\documentclass[aspectratio=169]{beamer}
\usepackage{utfpr_estilo}

\nomeTopico{Título}
\nomeEmail{e-mail@doauthor}
\nomeOrientador{Prof. Dr. Fábio Engel de Camargo}

\author{Nome do autor}

\barraInferior{TCC2 - UTFPR-TD/COTSI}


\begin{document}	
\frame{
	\titlepage
}
	
\begin{frame}[fragile]
	\frametitle{Sumário}
		\centering
		\tableofcontents
\end{frame}
	
\section{Introdução e Contextualização}
	
\begin{frame}[fragile]
	\frametitle{\secname}
	\begin{itemize}
		\item Tema: Apresente o tema e sua importância.
		\item Contexto: Contextualize o assunto dentro do cenário atual ou área de estudo.
	\end{itemize}
\end{frame}


\section{Problema de Pesquisa}

\begin{frame}[fragile]
	\frametitle{\secname}
	\begin{itemize}
		\item Apresentação do problema: Qual a pergunta principal que sua pesquisa busca responder? Deve ser clara, específica e passível de investigação.
		\item Mostre a relevância prática ou teórica do problema.
		\item Seja objetivo e claro para não gerar dúvidas.
	\end{itemize}
\end{frame}


\section{Objetivos}

\begin{frame}[fragile]
	\frametitle{\secname}
	\begin{itemize}
		\item Objetivo geral: O que você pretende alcançar com sua pesquisa de forma ampla? Geralmente, é um objetivo único e abrangente.
		\item Objetivos específicos: Quais são os passos ou as etapas que você precisa cumprir para atingir o objetivo geral? Mantenha os objetivos curtos e alinhados ao problema apresentado.
		\begin{itemize}
			\item Use verbos no infinitivo (analisar, descrever, identificar, propor, etc.).
		\end{itemize}
		\item \textbf{Importante}: Use os mesmos objetivos do documento entregue para a banca.
	\end{itemize}
\end{frame}

\section{Fundamentação Teórica}

\begin{frame}[fragile]
	\frametitle{\secname}
	\begin{itemize}
		\item Apresente os principais autores ou teorias que embasam o estudo.
		\item Conecte a teoria ao seu problema de pesquisa.
		\item Seja breve, foque no essencial para a banca entender o contexto.
	\end{itemize}
\end{frame}

\section{Metodologia}

\begin{frame}[fragile]
	\frametitle{\secname}
	\begin{itemize}
		\item Explique como o estudo foi realizado (métodos, técnicas ou procedimentos). 
		\item Destaque materiais, ferramentas ou amostra utilizada.
		\item Seja claro e objetivo para que qualquer pessoa possa reproduzir o estudo.
	\end{itemize}
\end{frame}

\section{Resultados}

\begin{frame}[fragile]
	\frametitle{\secname}
	\begin{itemize}
		\item Mostre os principais achados do trabalho.
		\item Use gráficos, tabelas ou imagens para ilustrar dados.
		\item Explique o significado dos resultados.
	\end{itemize}
\end{frame}

\section{Considerações Finais e Trabalhos Futuros}

\begin{frame}[fragile]
	\frametitle{\secname}
	\begin{itemize}
		\item Apresente as conclusões principais do estudo.
		\item Relacione as conclusões com os objetivos e o problema.
		\item Indique limitações do trabalho e possíveis pesquisas futuras.
	\end{itemize}
\end{frame}

\end{document}