\documentclass[aspectratio=169]{beamer}
\usepackage{utfpr_estilo}

\nomeTopico{Título}
\nomeEmail{e-mail@doauthor}
\nomeOrientador{Prof. Dr. Fábio Engel de Camargo}

\author{Nome do autor}

\barraInferior{TCC1 - UTFPR-TD/COTSI}


\begin{document}	
\frame{
	\titlepage    
}
	
\begin{frame}[fragile]
	\frametitle{Sumário}
	%descomente \begin{multicols}{2} e \end{multicols} para ter apenas uma coluna.
	\begin{multicols}{2}
		\centering
		\tableofcontents
	\end{multicols}
\end{frame}
	
\section{Introdução e Contextualização}
	
\begin{frame}[fragile]
	\frametitle{\secname}
	\begin{itemize}
		\item Contexto: 
		\begin{itemize}
			\item Apresente a área de estudo ao qual seu trabalho está inserido.
			\item Situar o contexto no qual o trabalho se insere
		\end{itemize}
	\end{itemize}
\end{frame}


\section{Problema de Pesquisa }

\begin{frame}[fragile]
	\frametitle{\secname}
	\begin{itemize}
		\item Apresentação do problema: Qual a pergunta principal que sua pesquisa busca responder? Deve ser clara, específica e passível de investigação.
		\item Justificativa: Por que é importante responder a essa pergunta? Qual a contribuição da sua pesquisa para a sua área ou para a sociedade ao solucionar esse problema?
	\end{itemize}
\end{frame}


\section{Objetivos}

\begin{frame}[fragile]
	\frametitle{\secname}
	\begin{itemize}
		\item Objetivo geral: O que você pretende alcançar com sua pesquisa de forma ampla? Geralmente, é um objetivo único e abrangente.
		\item Objetivos específicos: Quais são os passos ou as etapas que você precisa cumprir para atingir o objetivo geral? 
		\begin{itemize}
			\item Use verbos no infinitivo (analisar, descrever, identificar, propor, etc.).
		\end{itemize}
		\item \textbf{Importante}: Use os mesmos objetivos do documento entregue para a banca.
	\end{itemize}
\end{frame}

\section{Fundamentação Teórica Preliminar}

\begin{frame}[fragile]
	\frametitle{\secname}
	\begin{itemize}
		\item Conceitos chave: Apresente os principais conceitos e teorias que embasarão sua pesquisa.
		\item Pesquisas correlatas: Pode mencionar brevemente estudos similares que já foram realizados e como o seu se diferencia ou complementa.
		\item Autores e referências: Se houver, cite brevemente alguns autores importantes que você pretende usar e a relevância deles para o seu tema.
	\end{itemize}
\end{frame}

\section{Metodologia}

\begin{frame}[fragile]
	\frametitle{\secname}
	\begin{itemize}
		\item Delineamento da pesquisa:
		\begin{itemize}
			\item O que você vai fazer?
			\item Como vai fazer? Quais técnicas, ferramentas ou passos?
			\item Onde ou com quem? 
		\end{itemize}		
	\end{itemize}
\end{frame}

\section{Cronograma}

\begin{frame}[fragile]
	\frametitle{\secname}
	\begin{itemize}
		\item Apresentar o cronograma.
	\end{itemize}
\end{frame}

\section{Considerações Finais}

\begin{frame}[fragile]
	\frametitle{\secname}
	\begin{itemize}
		\item Reafirmação da relevância do problema e objetivos.
		\item Contribuições esperadas da pesquisa.
		\item Reforço da viabilidade e seriedade do projeto.
	\end{itemize}
\end{frame}

\end{document}