%%%% RESUMO
%%
%% Apresentação concisa dos pontos relevantes de um texto, fornecendo uma visão rápida e clara do conteúdo e das conclusões do
%% trabalho.

\begin{resumoutfpr}%% Ambiente resumoutfpr
O resumo deve ressaltar de forma sucinta o conteúdo do trabalho, incluindo justificativa, objetivos, metodologia, resultados e conclusão. Deve ser redigido em um único parágrafo, justificado, contendo de 150 até 500 palavras. Evitar incluir citações, fórmulas, equações e símbolos no resumo. A referência no resumo é elemento opcional em trabalhos acadêmicos, sendo que na UTFPR adotamos por não incluí-la nos resumos contidos nos próprios trabalhos. As palavras-chave e as keywords são grafadas em inicial minúscula quando não forem nome próprio ou nome científico e separados por ponto e vírgula.
\end{resumoutfpr}
