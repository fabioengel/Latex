\documentclass[a4paper,12pt]{coursepaper}
\usepackage{UTFPR}


\begin{document}
\nomeProfessor{Fábio Engel de Camargo}
\nomeCurso{Sistemas Para Internet}
\nomeDisciplina{Fundamentos de Redes de Computadores}
\dataDaProva{04/06/2025}
\tipoAvaliacao{Avaliação 1 (P1)}
\periodo{2}
\info



\begin{enumerate}[itemsep=0.5em]
	
	% 1 A
	\item  Qual é o principal objetivo da camada de enlace de dados em redes de computadores?
	\begin{description}[itemsep=-0.5em]
		\item[a)] Transformar um canal de transmissão bruto em uma linha que pareça livre de erros de transmissão não detectados para a camada de rede.
		\item[b)] Fornecer os requisitos para transportar pelo meio físico de rede os bits que formam o pacote da camada de Enlace de Dados.
		\item[c)] Gerenciar o endereçamento lógico e o roteamento de pacotes entre diferentes redes.
		\item[d)] Oferecer serviços de comunicação para aplicações de usuário, como e-mail e web.
		\item[e)] Dividir as informações em segmentos e garantir a entrega fim-a-fim.
	\end{description}
	
	% 2 C
	\item  Qual é a principal diferença entre um sinal analógico e um sinal digital?
	\begin{description}[itemsep=-0.5em]
		\item[a)] Sinais analógicos são usados apenas para voz, enquanto sinais digitais são usados para dados.
		\item[b)] Sinais analógicos podem ser transmitidos sem fio, enquanto sinais digitais exigem cabos.
		\item[c)] Um sinal analógico tem infinitamente muitos níveis de intensidade ao longo de um período de tempo, enquanto um sinal digital pode ter apenas um número limitado de valores definidos.
		\item[d)] A amplitude máxima de um sinal é a principal característica que os diferencia.
		\item[e)] Sinais digitais não sofrem atenuação, ao contrário dos analógicos.
	\end{description}
	
	
\end{enumerate}

\end{document}

